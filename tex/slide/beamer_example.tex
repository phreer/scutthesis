\section{\LaTeX-Beamer 排版说明}
\subsection{列表}
\subsubsection{无序列表}
\begin{frame}{无序列表}
同普通的 \LaTeX 文档一样,使用 \lstinline{itemize} 环境来得到无序列表:
\begin{itemize}
  \item 这是一个无序列表。
  \item 这是一个无序列表。
  \item 这是一个无序列表。
\end{itemize}
\end{frame}


\subsubsection{有序列表}
\begin{frame}{有序列表}
使用环境 \lstinline{enumerate} 创建有序列表, 使用方法无序列表类似。

\begin{enumerate}
  \item 这是一个有序列表。
  \item 这是一个有序列表。
  \item 这是一个有序列表。
\end{enumerate}

\end{frame}

\subsubsection{描述性列表}
\begin{frame}
使用环境 \lstinline{description} 可创建带有主题词的列表,条目语法是 
\lstinline{\item[主题] 内容}。
\begin{description}
    \item[主题一] 详细内容
    \item[主题二] 详细内容
    \item[主题三] 详细内容 \ldots
\end{description}
遗憾的是, 目前无法正常使用行内列表, 即以上几种列表的 * 版本. 另外也无法
使用自定义样式列表.

\end{frame}

\subsection{数学排版}
\label{sec:matheq}

\subsubsection{公式排版}
\label{sec:eqformat}
\begin{frame}{公式排版}
这里有举一个长公式排版的例子,来自 \href{http://www.tex.ac.uk/tex-archive/info/math/voss/mathmode/Mathmode.pdf}{《Math mode》}:

\begin{multline}
  \frac {1}{2}\Delta (f_{ij}f^{ij})= \\
  2\left (\sum _{i<j}\chi _{ij}(\sigma _{i}-
    \sigma _{j}) ^{2}+ f^{ij}\nabla _{j}\nabla _{i}(\Delta f)+\right .\\
  \left .+\nabla _{k}f_{ij}\nabla ^{k}f^{ij}+
    f^{ij}f^{k}\left [2\nabla _{i}R_{jk}-
      \nabla _{k}R_{ij}\right ]\vphantom {\sum _{i<j}}\right )
\end{multline}
\end{frame}

\subsubsection{定理环境}
\begin{frame}{定理环境}{留数定理}
这是一个定理的例子:
\begin{thm}[留数定理]
\label{thm:res}
  假设$U$是复平面上的一个单连通开子集,$a_1,\ldots,a_n$是复平面上有限个点,$f$是定义在$U\backslash \{a_1,\ldots,a_n\}$上的全纯函数,
  如果$\gamma$是一条把$a_1,\ldots,a_n$包围起来的可求长曲线,但不经过任何一个$a_k$,并且其起点与终点重合,那么:

  \begin{equation}
    \label{eq:res}
    \ointop_{\gamma}f(z)\,\mathrm{d}z = 2\uppi\mathbf{i}\sum^n_{k=1}
    \mathrm{I}(\gamma,a_k)\mathrm{Res}(f,a_k)
  \end{equation}

  如果$\gamma$是若尔当曲线,那么$\mathrm{I}(\gamma, a_k)=1$,因此:

  \begin{equation}
    \label{eq:resthm}
    \ointop_{\gamma}f(z)\,\mathrm{d}z = 2\uppi\mathbf{i}\sum^n_{k=1}
    \mathrm{Res}(f,a_k)
  \end{equation}

  % \oint_\gamma f(z)\, dz = 2\pi i \sum_{k=1}^n \mathrm{Res}(f, a_k ).

  在这里,$\mathrm{Res}(f, a_k)$表示$f$在点$a_k$的留数,$\mathrm{I}(\gamma,a_k)$表示$\gamma$关于点$a_k$的卷绕数。
  卷绕数是一个整数,它描述了曲线$\gamma$绕过点$a_k$的次数。如果$\gamma$依逆时针方向绕着$a_k$移动,卷绕数就是一个正数,
  如果$\gamma$根本不绕过$a_k$,卷绕数就是零。
\end{thm}
\end{frame}

\subsubsection{证明环境}
\begin{frame}{留数定理的证明}
\begin{proof}
  首先,由……

  其次,……

  所以……
\end{proof}

上面的公式例子中,有一些细节希望大家注意。微分号d应该使用“直立体”也就是用mathrm包围起来。
并且,微分号和被积函数之间应该有一段小间隔,可以插入 \lstinline{\\} 得到。
斜体的$d$通常只作为一般变量。
i,j作为虚数单位时,也应该使用“直立体”为了明显,还加上了粗体,例如 \lstinline{\\mathbf\{i\}}。斜体$i,j$通常用作表示“序号”。
其他字母在表示常量时,也推荐使用“直立体”譬如,圆周率$\uppi$(需要upgreek宏包),自然对数的底$\mathrm{e}$。
不过,我个人觉得斜体的$e$和$\pi$很潇洒,在不至于引起混淆的情况下,我也用这两个字母的斜体表示对应的常量。

\end{frame}